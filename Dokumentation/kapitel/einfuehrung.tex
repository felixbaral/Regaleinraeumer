\section{Einführung}
\subsection{Modulbeschreibung}
Dieses Dokument beschreibt die Anforderungen und Implementierungsdetails an eine Echtzeitsystemanwendung, welche das Abschlussprojekt des Moduls Echtzeitbetriebssysteme der EAH Jena darstellt. Dieses Dokument richtet sich dabei nach den Vorgaben für die Software Requirements Specification (SRS) nach dem \href{https://de.wikipedia.org/wiki/Software_Requirements_Specification}{IEEE Standard 830-1998}. Der Modulverantwortliche ist Prof Dr. Oliver Jack aus dem Fachbereich Elektrotechnik und Informationstechnik.\\

\subsection{Einstieg in die Projektphase}
Die Arbeit an der Programmumsetzung wurde in die Teilbereiche Hochregal-Steuerung, Simulation, Integritätsprüfung, Benutzereingabe und Visualisierung aufgeteilt, um diese Bereiche allein bzw. in Gruppen parallel  abarbeiten zu können.\\
Dabei ergab sich folgende Aufteilung:
\begin{itemize} 
	\item Hochregal-Steuerung $\rightarrow$ Michael Thomas, Simon Weitzel
	\item Simulation $\rightarrow$ Felix Baral-Weber, Andreas Glatz
	\item Integritätsprüfung und Benutzereingabe $\rightarrow$ Simon Weitzel, Michael Thomas
	\item Visualisierung $\rightarrow$ Andreas Glatz, Felix Baral-Weber
\end{itemize}

Desweiteren wurden Hauptverantwortliche für alle Teilbereich des Projekt festgelegt:\\

\begin{itemize} 
	\item Michael Thomas: Programmumsetzung
	\item Andreas Glatz: Tests
	\item Simon Weitzel: Strukturierte Analyse
	\item Felix Baral-Weber: Latex-Template und Funktionale Anforderung
\end{itemize}