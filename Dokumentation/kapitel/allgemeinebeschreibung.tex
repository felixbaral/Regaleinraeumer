\section{Allgemeine Beschreibung}
\subsection{Aufgabenstellung und Umsetzungsansatz}
Die Aufgabestellung des Semesterprojektes ist es eine Hochregallagersimulation unter zuhilfenahme eines Echtzeitbetriebssystems zu erstellen. Diese soll das gesamte System mit einer statisch gewählten Regaldimension sowohl steuern als auch simulieren.
Das Simulationmodell umfasst ein Regallagersystem mit $5*10$ Regalfächern, ein Ein- und ein Ausgabeslot und einen Turm zum be- und entladen. Dieser Turm ist auf einer Achse montiert auf welcher er sich in X-Richtung bewegen kann, desweiteren kann der Ausleger am Turm in Y-Richtung herauf und herab und in Z-Richtung rein und raus gefahren werden. Diese Bewegungen und die daraus resultierende Position des Turms werden von Tastern auf der Schiene ermittelt.\\
Auf der X-Achse und auf der Y-Achse sind jeweils 10 Taster und auf der Z-Achse 3.

Dem Anwender steht als Eingabe- und Ausgabemedium die Konsole zur Verfügung.
\newline\newline
Ihm stehen folgende Befehle zur Verfügung:
\begin{itemize} 
	\item getspace $\rightarrow$ Zeigt aktuelle Belegung des Hochregallagers an
	\item vsetspace[x][y] $\rightarrow$ Definiert einen Platz als schon belegt
	\item clearspace[x][y] $\rightarrow$ Definiert einen Platz als frei
	\item insert[x][y] $\rightarrow$ Holt ein Klötzchen vom Eingabe-Slot und legt es an gewünschter Position im Hochregallager ab
	\item remove[x][y] $\rightarrow$ Holt ein Klötuchen von der gewünschten Position ab und legt es an den Ausgabe-Slot
\end{itemize}

Sollte ein Befehl nicht möglich sein, da zum Beispiel ein Hochregallagerplatz oder der Ausgabe-Slot bereits belegt ist, wird der Anwender durch eine Fehlermeldung in der Konsole darauf aufmerksam gemacht und der Befehl nicht ausgeführt.

Die Visualisierung des Hochregallagers erfolgt ebenfalls in der Konsole, in welcher sowohl das Hochregal und dessen Belegung als auch die Position des Turms und dessen Auslegers, durch ASCI-Zeichen stilisiert dargestellt werden.
Dabei liegt der Ursprung der Koordinaten und somit der Regalplatz (0,0) in der linken unteren Ecke.
Die Darstellung des Ein- und Ausfahren des Auslegers wird unterhalb dargestellt. Dabei stellt die linke Position den zum Ein-/Ausgabe-Slot gefahrenen Arm da, die rechte Position die in das Regal hereingefahrene.
\todo{Bild der Konsole einfügen}

\subsection{Entwicklungsumgebung und Programmierung}
Als Entwicklungs- und Testumgebung wurde Windriver Workbench 3.3 genutz und die Programmiersprache C verwendet.
Es wurden für das Echtzeitbetriebssystem Vxworks Libaries inkludiert.


